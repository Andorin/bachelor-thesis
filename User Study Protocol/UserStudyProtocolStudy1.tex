\documentclass[10pt,a4paper]{article}
\usepackage[utf8]{inputenc}
\usepackage{amsmath}
\usepackage{amsfonts}
\usepackage{amssymb}

\begin{document}
\begin{center}
\textbf{User Study Procedure Protocol}
\end{center}
\section{Welcome the user and tell her the purpose of the study.}

\begin{itemize}
\item we want to test the physical limitations of our 2D touch prototype
\item how reliable is our prototype in different conditions (curvature, underlying softness and overlying materials with different frictions
\item the user has to perform different gestures in each condition several times
\end{itemize}

\section{Introduce the user to the prototype.}
\begin{itemize}
\item show the user the set of gestures and demonstrate some of them showing the user the result on the screen
\item let the user try some of the gestures on the flat surface and show him the results.
\item ask the user about his/her first impression, make notes
\item prepare the first condition of the user study
\end{itemize}

\section{Performing the User Study}
\begin{itemize}
\item (Curvature) 3 x (Softness) 3 x (Material) 4 x (gestures) 16 x (repetition) 3 = 1728 trials
\item after a condition has been set up the user performs all  16 gestures counter-balanced while being video-recorded
\item the output is hidden from the user, only we observe it, screen is captured
\item we can provide some tips when we discover that too less pressure is applied or the gesture is performed too fast
\item the preparation for each condition gives the user time to relax
\item ask the participant how easy the gesture was to perform using a 7 item likert-scale
\item before every condition ask the user if she needs some more rest
\end{itemize}

\section{After the User Study}
\begin{itemize}
\item ask: the participant how he/she feels especially in the dominant index finger
\item ask: can you imagine to use this technology on daily basis
\item ask: thoughts about our prototype
\item thank the user for participation 
\end{itemize}
\end{document}