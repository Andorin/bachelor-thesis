%!TEX root = ./main.tex
%
% This file is part of the i10 thesis template developed and used by the
% Media Computing Group at RWTH Aachen University.
% The current version of this template can be obtained at
% <http://www.media.informatik.rwth-aachen.de/karrer.html>.

\chapter{Introduction}
\label{introduction}
 
\mnote{Motivation and well known wearables}Electronics are getting smaller, lighter and more powerful every year and we reached a point where they have actually become wearable. Therefore the research in the field of wearable computing increased over the last decade. The ultimate goal is to make life easier and more comfortable by integrating controls and sensors even more in our daily life. Health tracking devices are the leading wearables at the moment. Smartwatches are even capable of various smartphone features such that it is less often necessary to take your smartphone out of your pocket. \\ \\
\mnote{Smart clothing}One of the first contributions to the field of wearables were made by \cite{post1997smart}. They emebedded easy to build, washable textile based sensors, buttons, and switches into a jacket. \cite{Rantanen:2002:SCP:594096.594098} integrated a computer including  screen and battery into an arctic suite to provide the wearer with information about the surrounding conditions, their location, and controls for the integrated heating system. \cite{Brewster:2003:MEI:642611.642694} investigated the opportunities of eyes-free hand gestures on a PDA attached to the waist, supported by audio feedback.\\ \\
The focus of this thesis lies on wearable textile input devices that can be integrated into everyday clothing. Devices for eyes-free interaction are primary designed for a mobile context where the visual channel is occupied by the environment. While driving a car, changing the radio station, skipping a song, or answering a call activating the eyes-free feature on your devices is one application for a wearable touchpad on your thigh. Interacting with a textile sensor in your sleeve, for example, will reduce the division of attention.\\
Various approaches were presented aiming to create wearable input devices over the last decades. The most used technique for sensing a touch is capacitive touch  sensing. \citep{Holleis:2008:ECT:1409240.1409250} built textile prototypes by sewing conductive thread into fabric creating touch sensing buttons. The buttons are discrete input elements and not continuous. However, their research resulted in several guidelines applicable for this field in general. \\ \\
\mnote{Limitation and improvements}Most of the textile touchpads today are based on capacitive touch and rather prone to noise. The number of gestures they are able to distinguish reliably is quite limited due to body water. To improve their performance the influence of the human body has to be minimized by improved shielding techniques. These techniques, however, are not accessible today.  Therefore we use resistive technology in this thesis.\\ \\
\mnote{Our contributions}In this thesis we present our wearable, resistive textile 2D touchpad. We provide a detailed description how to build this sensor with low cost materials. We show that our prototype is able to reliably detect various unistroke gestures and evaluate its robustness under several conditions. The software for operating the sensor is provided in chapter~\ref{appendix}.
