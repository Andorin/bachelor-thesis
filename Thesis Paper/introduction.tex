%!TEX root = ./main.tex
%
% This file is part of the i10 thesis template developed and used by the
% Media Computing Group at RWTH Aachen University.
% The current version of this template can be obtained at
% <http://www.media.informatik.rwth-aachen.de/karrer.html>.

\chapter{Introduction}
\label{introduction}
 
\mnote{Motivation}Electronics are getting smaller, lighter and powerful every year and we reached a point where they have actually become wearable. Therefore the research in the field of wearable computing increased over the last decade. The ultimate goal is to make life easier and more comfortable by integrating controls and sensors even more in your daily life. Health tracking devices are the leading wearables at the moment. Smartwatches are even capable of various smartphone features such that it is less often necessary to take your smartphone out of your pocket. \\ \\
The focus of this thesis lies on wearable textile input devices for eyes-free interaction integrated into everyday clothing. Devices for eyes-free interaction are primary designed for a mobile context where the visual channel is occupied by the environment. Various approaches were presented aiming to create wearable input devices over the last decades.
\mnote{Wearables over the last years}The most used technique for sensing a touch is capacitive touch  sensing. \citep{Holleis:2008:ECT:1409240.1409250} built textile prototypes by sewing conductive thread in fabric creating touch sensing buttons. The buttons are discrete input elements and not continuous. However, their research on conductive resulted in several guidelines applicable for this field in general. 
\mnote{Limitation and improvements}Most of the touchpads so far are rather prone to noise
\mnote{Our contributions}In this thesis we present our novel wearable, bendable, washable, scalable textile 2D touchpad. Furthermore we provide a detailed description how to build this sensor with low cost materials. We show that our prototype is able to reliably detect various unistroke gestures and evaluate its robustness under several conditions. A link to software for operating the sensor is provided in the appendix.
