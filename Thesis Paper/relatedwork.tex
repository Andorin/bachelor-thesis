%!TEX root = ./main.tex
%
% This file is part of the i10 thesis template developed and used by the
% Media Computing Group at RWTH Aachen University.
% The current version of this template can be obtained at
% <http://www.media.informatik.rwth-aachen.de/karrer.html>.

\chapter{Related work}
\label{relatedwork}
%In this chapter we first describe the basic differences between the two most popular touch input technologies and which is best suited for wearable scenarios. Then we give an overview of the latest related research in wearable touch input devices. We will  focus on hardware prototypes and their underlying techniques. Wearable computation is a current field of research over the past years. Therefore a lot of different approaches were made to make textiles touch sensitive hence we only present some of them. An overview of the presented touch pad prototypes is shown in table\ref{table:overview}.
This chapter reviews related research in the area of interactive textile. We divided this review into two parts: interactive textile technology and textile touch pads. In the first part we give an overview of ways to integrate and activate textile in everyday objects. In the seconds part we look at research that investigated different techniques to fabricate textile surfaces that can detect user touches and gestures.

\section{General overview}
\subsection{Resistive vs. capacitive touch}
The two most popular touch input technologies are resistive and capacitive. Both serve the same purpose but the underlying principle differs making them more or less suited for wearable computing. 

\mnote{Characteristics of Capacitive Touch}Capacitve touch uses a non-conducting material with conductive material underneath. The capacitance of the human body changes the electrical field of the sensor which is measurable. The advantages of capacitive touch is the easy support for multi-touch input and today's touch screens only need a slight touch without force. The main disadvantage is the human body itself, since  it generates its own capacitive field which makes it hard to detect the intentional touch. This flaw is intensified by the body movement continuously changing the proximity between the sensor and the human body. Therefore complex isolation technique is required to isolate the sensor and the human body which is unfeasible for fast prototyping. \\ \\
\mnote{Characteristics of Resistive Touch}Resistive touch technology uses two separated layers of striped electrodes such that it is arranged to a matrix. The spacing material in ordinary resistive touch screens is either an air filled chamber or a non-conductive material which separates both layers while no external force is applied. Therefore one can operate it with a stylus or with gloves since no conductivity is required. On the one hand this solves the main disadvantage of the capacitive method regarding wearable computing, on the other hand it is still prone to deformation and thus to unintentional contacts.

\subsection{Textile interaction techniques}
\cite{Holleis:2008:ECT:1409240.1409250} presented several textile prototypes based on capacitive sensing. They embroidered conductive wires to  a phone case, a glove, and an apron resulting in small conductive buttons. Beside that they used conductive foil for buttons on an helmet as well. Their user study however only evaluated the apron with three different button layouts with different visibility. Based on the results they present guidelines for wearable controls such as locating and identifying controls must be quick and easy.
\\
\cite{Speir:2014:WRC:2628363.2634221} built two prototypes, a wristband and a glove. The wristband prototype uses a circular conductive fabric surrounded by resistive linqstat. A conductive finger cap connects both of them an generates a value which is used to determine the location and the movement of the touch. The glove works on the same principle. They evaluated their prototypes as remote controls for an iPod using one- and two-handed interaction and have found that the users have no clear preference. 

\mnote{Ubiquitous drums}A different application is presented by \cite{Smus:2010:UDT:1753846.1754094}. They used force-sensitive resistors and pull-down resistor circuits to sense percussive touch. They taped the sensor into the inside of a pair of jeans and to the sole of a shoe. They created a program that translates the sensor values to different parts of a drum. 

\mnote{Time Domain Reflectometry}
\citep{Wimmer:2011:MDT:2047196.2047264} created 13 prototypes based on time domain reflectometry (TDR). For this approach only one pair of wires is needed. The change in capacitance caused by conductive objects close to the pair of wire is measured and the location determined. Distance between the wires and their shape have significant influence on reliability. Their prototypes include stretchable, curved, and arbitrary shaped surfaces. They can sense touch at a distance up to 20m but TDR is prone to radio interference of mobile phones. 

\section{Textile touch pads}
\mnote{Pinstripe}Pinstripe is a continous textile input prototype created by \cite{Karrer:2010:PEC:1866218.1866255}. It detects pinching and rolling of clothing by connecting conductive thread sewn on it. However, it is an unidimensional input device and not a touch pad. Nevertheless, when they introduced the users to the sensor they intuitively expected a touch pad. \\
\mnote{Grabrics}Grabrics by \cite{hamdan} is a fold-based textile sensor that can detect the axis of a pinch and the displacement and direction of the user’s thumb over the fold. It, however, cannot detect comple gesture because of the limited resolution.\\
\mnote{GesturePad}\cite{Rekimoto:2001:962092} presented GesturePad which is a capacitive touch pad integrated in clothing. They propose slightly different architectures consisting of the upper fabric, receiver, transmitter, and a shield layer to reduce the influence of the human body. However, their work was not further evaluated. \\
\mnote{PocketTouch}Another capacitive approach was developed by \cite{Saponas:2011:PTC:2047196.2047235}. PocketTouch is an eyes-free, calibrateable capacitive touch pad which can sense the proximity of a finger through a wide range of fabrics. They used a touch sensor of a touch screen and attached it to a base which makes it not bendable. The reliability of PocketTouch was not further evaluated. \\
\mnote{FabriTouch}FabriTouch is a flexible, capacitive textile touch pad presented by \cite{Heller:2014:FEF:2634317.2634345}. It consists of lining, piezoresistive foil, spacing mesh, conductive fabric, and outer garment together integrated into a pair of trousers. FabriTouch is able to reliably detect simple swipe gestures on rigid surfaces rather than on the human thigh. Also movement has a negative impact on the performance of the sensor.
\begin{table}
\begin{tabular}{ | c | c | p{3.8cm}|}
\hline
  Prototype & Touch technology & Gesture detection \\
  \hline
   Pinstripe & capacitive  & detects size of pinch and movement of pinch in 1D \\
   \hline
  GesturePad & capacitive &  not tested (theoretically able to detect 2D gestures) \\
  \hline
  Pocket touch & capacitive & multistroke gestures using N\$ by \cite{anthony2012n} \\
  \hline
  FabriTouch & capacitive & swipes in 2D \\
  \hline
  Grabrics & resistive & detects axis of fold and movement of pinch in 2D \\ 
  \hline
\end{tabular}
\caption{Current textile touch pad technologies.}
 \label{table:overview}
\end{table}

\mnote{Going for Resistive Touch}After taking all characteristics into account we decided to go for the resistive touch technology, because we can drop all considerations of capacitive noise caused by the human body. 