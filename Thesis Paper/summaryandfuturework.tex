%!TEX root = ./main.tex
%
% This file is part of the i10 thesis template developed and used by the
% Media Computing Group at RWTH Aachen University.
% The current version of this template can be obtained at
% <http://www.media.informatik.rwth-aachen.de/karrer.html>.

\chapter{Summary and future work}
\label{summaryandfuturework}

In this chapter we will give a summary of our contribution to the field of human computer interaction. Reviewing the hardware and its capabilities will point us to constitutive ideas for future work.

\section{Summary and contributions}
\label{summaryandfuturework.summary}
We presented a 2D textile touchpad for eyes free interaction with novel properties compared to the latest prototypes in this field. It is based on the simple principles of resistive touch technology which has some advantages over capacitive technology when it comes to wearable touch sensing. We presented several related touch pad prototypes in chapter~\ref{relatedwork}. Most of them use capacitive touch and the gestures they are able to detect are rather limited due to the noise generated by the human body. Our prototype only yields a touch if the layers are physically connected. \emph{Grabrics} uses resistive touch as well but the interaction design differs from simple 2D touch. 
\\ \\
In chapter~\ref{Hardware Prototype and Software Development} we described how we built our prototype with low cost materials. The touch pad itself is made out of textiles only making it bendable and breathable, but limit the stretchability of the sensor at the same time. We explained step by step how to attach the pinstripe fabric layers to the spacing material, such that everyone is able to rebuild it in short time. The necessary code is linked in the \ref{appendix}ppendix. Our prototype is easy scalable and is only limited in the number of pins of the used microcontroller. Although we made our prototypes equilateral, it is simply possible to give it any rectangular size. We want to note that a lot of testing of the 14 by 14 was done prior to the user study. 
\\
Furthermore we explained the software for our sensor to detect simple unistroke mark-based gestures using our own recognizer. Then we went one step further and even recognized more complex unistroke free-form gestures. This is the first full textile touch pad being able to do that consistently. An informal user study, presented in chapter~\ref{evaluation}, has shown that. 
We let two participants test the prototype under multiple conditions to evaluate the physical limitations of the sensor. We found that there is a learning effect since we observed better results from the more experienced user. 
\\
Beside that we found that the 1\$ recognizer is not optimal for sensors with a rather low resolution of 14 by 14. Curvature seems to have only significant impact on the overall performance which makes the thigh best suited additional to the fact that both participants prefer jeans fabric for interacting with the sensor. Although both participants liked operating the sensor, both agree that it gets unpleasant over time it it would be best suited for occasional use.

\section{Future work}
\label{summaryandfuturework.futurework}
\index{future work|}
The most immediate step would  be making the sensor actually wearable by integrating it into everyday clothing. This yields new challenges beside recognizing 2D touch. The wiring of sensor and micrcontroller and the power supply should be imperceptible. Furthermore the data processing and gesture recognition should be ported to the microcontroller. 
\\ \\
A number of embedded prototypes could be built with more different fabrics used in today's clothing. Then series of user studies could be conducted to test the performance of the sensor in daily use.
\index{future work|)}
