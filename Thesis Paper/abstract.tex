%!TEX root = ./main.tex

% This file is part of the i10 thesis template developed and used by the
% Media Computing Group at RWTH Aachen University.
% The current version of this template can be obtained at
% <http://www.media.informatik.rwth-aachen.de/karrer.html>.

\loadgeometry{myAbstract}

\chapter*{Abstract\markboth{Abstract}{Abstract}}
\addcontentsline{toc}{chapter}{\protect\numberline{}Abstract}
\label{abstract}

This bachelor thesis describes the development of a wearable 2D textile touchpad able to be embedded into everyday clothing. It is a wearable, bendable, washable, and breathable touch sensor designed for eyes-free interaction. The touchpad is basically composed of 3 textile layer, two perpendicular layers of fabric with conductive stripes, separated by a non conductive material. All materials used for manufacturing are low cost and lightweight. The prototype is based on resistive touch, hence it is limited to sensing touch at a single location. The touch coordinates are continuously sent to a gesture recognizer allowing the prototype to detect unistroke gestures. With these gestures can be used to control several applications in a mobile scenario. The physical limitations are evaluated in an informal user study by testing the performance under various conditions. 

\chapter*{\"Uberblick\markboth{\"Uberblick}{\"Uberblick}}
\addcontentsline{toc}{chapter}{\protect\numberline{}\"Uberblick}
\label{ueberblick}

In dieser Bachelorarbeit wird die Entwicklung eines tragbaren, biegsamen, waschbaren und atmungsaktiven 2D Touchpads beschrieben. Der Sensor besteht aus aus zwei Schichten Stoff, der mit  leitfähige Bahnen versehen ist, und Schaumstoff, der die beiden orthogonalen Stoffschichten trennt. Alle benutzte Materialien sind leicht und kostengünstig. Der Prototyp basiert auf der resistiven Methodik eine Berührung zu erkennen. Das heißt, dass der Sensor auf eine einzige Position beschränkt ist. Diese Positionen werden kontinuierlich an ein Programm, dass aus den einzelnen Punkten Gesten erkennt, geschickt. Damit können dann diverse Anwendungen insbesondere im mobilen Bereich gesteuert werden. Außerdem haben haben Benutzer den Prototypen unter verschiedenen Bedingungen getestet, um die physikalischen Einschränkungen zu bestimmen.

\loadgeometry{myText}
