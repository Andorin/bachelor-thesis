,%!TEX root = ./main.tex
%
% This file is part of the i10 thesis template developed and used by the
% Media Computing Group at RWTH Aachen University.
% The current version of this template can be obtained at
% <http://www.media.informatik.rwth-aachen.de/karrer.html>.

\chapter{Hardware Prototype and Software Development}
\label{Hardware Prototype and Software Development} 
In this chapter we first describe the basic differences between the two most popular touch input technologies and which is best suited for wearable scenarios. Then we proceed with showing several iterations of the hardware prototypes using pinstripe and the corresponding software. Furthermore we describe the disadvantages and improvements of each iteration. The results of two user studies are discussed afterwards. 
\section{Resistive vs. Capacitive Touch}
\mnote{Advantages and Disadvantages of Resistive and Capacitive Touch Technology}
The two most popular touch input technologies are resistive and capacitive. Both serve the same purpose but the underlying principle differs making them more or less suited for wearable computing. 
\subsection{Capacitive Touch}
\mnote{Characteristics of Capacitive Touch}
\todo{find simple explanation of capacitive touch.}

The advantages of capacitive touch is the easy support for multi-touch input and the almost no need to press down the area of the sensor. The main disadvantage is the human body itself, since  it generates its own capacitive field which makes it hard to detect the intentional touch. This flaw is intensified by the movement of the textile due to body movement. \textbf{TODO: source! Especially for earlier findings why it is worse than resistive}

\subsection{Resistive Touch}
\mnote{Characteristics of Resistive Touch}
Resistive touch technology uses two separated layers of striped electrodes such that it is arranged to a matrix. The spacing material in ordinary resistive touch screens is either an air filled chamber or a non-conductive material which separates both layers while no external force is applied. Therefore one can operate it with a stylus or with gloves since no conductivity is required. On the one hand this solves the main disadvantage of the capacitive method regarding wearable computing, on the other hand it is still prone to deformation and thus to unintentional contacts.

\mnote{Going for Resistive Touch}
After taking all characteristics into account we decided to go for the resistive touch technology, because we can drop all considerations of capacitive noise caused by the human body. 

\section{System Design}
\mnote{MSP430 for first prototype}
\todo{exact specification of pinstripe and micro-controller und den gripper probe Klemmkabeln!!}
All prototypes presented here are using pinstripe, a fabric with separated conductive lines. For the first prototype we were using the Texas Instruments MSP430G2452 micro-controller. Each row and column of the pinstripe has to be connected to a \emph{digitalRead} pin of the micro-controller. The MSP430 controller has 16 of these pins but only 14 can be used since two pins are used for serial communication. This results in a matrix resolution of 7 by 7 at maximum. 
\medskip
\myDefBox{Resolution in pinstripe context}{When speaking of a certain resolution of our prototype, we talk about the number of connected rows and columns. Since the pinstripe fabric is of a static size, the higher the resolution the larger the prototype gets. }

We are using the TI EK-TM4C1294XL for the advanced prototypes. This board has the ability to connect more than 40 pins for \emph{digitalRead} to operate a 20 by 20 pinstripe matrix. The board is connected to a Computer via USB for serial communication.\\

\mnote{Programming Environment: Energia IDE and Processing to gather, send and process input data}
For programming the micro-controller we are using the \href{http://energia.nu}{Energia IDE}\footnote{http://energia.nu}. It is an easy to use IDE to upload programs to the TI micro-controller. The micro-controller itself is solely responsible for sending the data of the sensor to the computer via serial communication. Meaning that it tests a pin against ground for each other line and column. A \emph{1} is written to the serial-port when it is connected to another line or column and \emph{0} otherwise. This is done for each pin where \emph{numberOfPintripes} is the number of all lines and columns. For each prototype an integer array is declared and can easily be commented and uncommented depending on the prototype. The where the pin numbers are sorted such that the first pins correspond to the x-axis and the last pins to the negative y-axis. After all pins were tested we send a line-break to determine the end of the current input.\\  
\newline
\lstset{frame=tb,
  language=Java,
  aboveskip=3mm,
  belowskip=3mm,
  showstringspaces=false,
  columns=flexible,
  basicstyle={\small\ttfamily},
  numbers=none,
  numberstyle=\tiny\color{gray},
  keywordstyle=\color{blue},
  commentstyle=\color{dkgreen},
  stringstyle=\color{magenta},
  breaklines=true,
  breakatwhitespace=true,
  tabsize=3
}
\begin{lstlisting}
void loop() {
  for (int i = 0; i < numberOfPinstripes; i++) {
    pinMode(array[i], INPUT_PULLUP);
    delay(2);
    state = digitalRead(array[i]) ? '0' : '1';
    pinMode(array[i], OUTPUT);
    digitalWrite(array[i], LOW);
    Serial.print(state);
  }
  Serial.println();
  delay(3);
}
\end{lstlisting}
We are using \href{http://processing.org}{Processing}\footnote{http://processing.org}, a Java based IDE, to structure the input stream from the micro-controller for further analyses. This includes several programs which either displays the raw touch points for debugging purpose or filters and interprets the sensor data. The changes of software are described along with each hardware iteration. 

\section{Early Testing}
\mnote{better note something here}
After deciding to go for the resistive approach, the essential challenge is to find a spacing material with certain characteristics. The material should
\begin{itemize}
\item be flexible by means of being wearable.
\item reliably separate the pinstripe fabric while no touch is intended.
\item concede rather easily when intended force is applied.
\end{itemize}
We start by trying some leftovers from recent research projects. 
\todo{Find names of the fabrics and materials} 
Then we place the materials between the pinstripe fabric to examine different behaviors. We glued both layers of the pinstripe fabric to sheets of paper to eliminate stretching and curling of the fabric. Some of the the materials are cut with a laser-cutter. We cut equidistant circular holes to provide space for the pinstripe layers to connect. We attached 4 by 4 pinstripes to the MSP430 micro-controller. For displaying where a touch is present, we created a simple program with Processing.
\todo{insert picture here}
\mnote{Combine materials to improve results}Almost all available materials are not well suited for our purpose, since they are either too thick, thin or stiff. When combining up to two materials, which are leading to permanent contacts on their own, the results are improving. 
\section{First Iteration of the Prototype}



\chapter{Evaluation}
\mnote{Testing the 14 by 14 prototype}
In this chapter we will take a closer look at the performance of the 14 by 14 prototype. Since the prototype is designed as a wearable, we are interested in its behavior under certain changing conditions. We conducted a user study to test the physical limitations of our prototype. 

\section{Physical Limitation Study}
\mnote{Independent variables: friction, softness, looseness, and curvature}
The human body is in motion almost all the time and the clothes are not fixed to it. This \emph{looseness} and the changing subsurface are variables that may influence the performance of our prototype. Another variable is the \emph{friction} of the overlaying material. Depending on the fabric and method of fashioning, it can, more or less likely, happen that the user slips of the touch-sensing area, or experiences an unpleasant feeling in the operating finger. Furthermore the \emph{softness} of the underlying surface may influence the performance of our prototype. The human body has different softness almost everywhere. The amount of muscles, adipose tissue, and so forth also differs from human to human. This affects the pressure needed by the user in the first place. Then there are the different levels of curvature. 

\section{Study Design}

\section{Study Procedure}
After the user arrived we introduced her to her to our prototype. We explained the basic functionality and demonstrated how the output looks like. Then we let the user test the eight gestures and some freestyle strokes. This was done without foam or any additional fabric. We pointed out that a certain amount of pressure is essential for our prototype to recognize the touch. When they felt familiar enough, about 2 minutes of testing, we prepared the first condition. \\

\section{Participants}

\section{Results and Analysis}

